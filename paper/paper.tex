\documentclass{article}

\usepackage[spanish]{babel}
\usepackage{subcaption}
\usepackage{graphicx}
\usepackage{amsmath}
\usepackage{pgffor,pgf}
\usepackage{bm}
\usepackage{tikz}
\usetikzlibrary{babel}

\captionsetup[subfigure]{subrefformat=simple,labelformat=simple}
\renewcommand\thesubfigure{(\arabic{subfigure})}


\title{Backpropagation}
\author{Aurora Zuoris \\ \texttt{aurora.zuoris101@alu.ulpgc.es}}

\begin{document}

\maketitle



\section{Diferenciación automática}

\subsection{Operaciones matemáticas}

\subsubsection*{Adición}

\newpage

\subsection{Funciones de activación}

\subsubsection*{Softmax}

\begin{align*}
	\bm{s} &= \text{Softmax}(\bm{x}) \\
	g &= f(\bm{s})
\end{align*}

El softmax es una de las funciones más complicadas a diferenciar
dado que cada elemento del vector de entrada es relacionado con
cada vector de salída.
Para resolver esto, hay que utilizar la regla de la cadena multivariable.

\def\empty{null}

\begin{figure*}[h]
\centering
\begin{subfigure}{0.45\linewidth}
\centering
\begin{tikzpicture}[
	value/.style={circle,draw},
	value \empty/.style={
		draw=none,
		execute at begin node=$\hdots$
	}
]
	\node[value] (g) at (0,6) {$g$};
	\foreach \i [count=\x] in {1,2,3,null,n} {
		\node[value,value \i/.try] (s-\i) at (\x-3,4) {\ifx\i\empty \else$s_\i$\fi};
	}

	\foreach \i [count=\x] in {1,2,3,null,n} {
		\node[value,value \i/.try] (x-\i) at (\x-3,2) {\ifx\i\empty \else$x_\i$\fi};
	}

	\foreach \i in {1,2,3,n}
		\draw[->] (g) -- (s-\i);
	\foreach \i in {1,2,3,n}
		\foreach \j in {1,2,3,n}
			\draw[->] (s-\i) -- (x-\j);
\end{tikzpicture}
\caption{grafo completo de dependencias}
\end{subfigure}
\hfill
\begin{subfigure}{0.45\linewidth}
\centering
\begin{tikzpicture}[
	value/.style={circle,draw},
	value \empty/.style={
		draw=none,
		execute at begin node=$\hdots$
	},
	label/.style={midway,fill=white,inner sep=1pt}
]
	\node[value] (g) at (0,6) {$g$};
	\foreach \i [count=\x] in {1,2,3,null,n} {
		\node[value,value \i/.try] (s-\i) at (\x*1.2-3*1.2,4) {\ifx\i\empty \else$s_\i$\fi};
	}

	\node[value] (x) at (0,2) {$x_i$};

	\foreach \i in {1,2,3,n}
		\draw[->] (g) -- (s-\i) node[label,pos=0.6] {$\frac{\partial g}{\partial s_\i}$};
	\foreach \i in {1,2,3,n}
			\draw[->] (s-\i) -- (x) node[label,pos=0.4] {$\frac{\partial s_\i}{x_i}$};
\end{tikzpicture}
\caption{grafo simplificado a una entrada}
\end{subfigure}
\end{figure*}

\begin{equation*}
	\frac{\partial g}{\partial x_i} = \sum_{j=1}^n \frac{\partial g}{\partial s_j} \frac{\partial s_j}{\partial x_i}
\end{equation*}

Esto se puede simplificar de forma matricial teniendo en cuenta que hay $n$ valores de $\frac{\partial g}{\partial s_j}$ y $n\times n$ valores de $\frac{\partial s_j}{\partial x_i}$,
tal que la primera se puede representar como el gradiente de $g$ respecto a $s$ y la segunda como la matriz jacobiana de $s$ respecto a $x$,
de forma que el producto matricial de estas dos nos da el gradiente con respecto a $\bm{x}$.

\renewcommand{\arraystretch}{1.8}

\def\one{1}

\begin{equation*}
	\frac{\partial g}{\partial \bm{x}} = 
	\frac{\partial \bm{s}}{\partial \bm{x}} \frac{\partial g}{\partial \bm{s}} =
	\begin{bmatrix}
		\frac{\partial s_1}{\partial x_1} & \frac{\partial s_2}{\partial x_1} & \cdots & \frac{\partial s_n}{\partial x_1} \\
		\frac{\partial s_1}{\partial x_2} & \frac{\partial s_2}{\partial x_2} & \cdots & \frac{\partial s_n}{\partial x_2} \\
		\vdots & \vdots & \ddots & \vdots \\
		\frac{\partial s_1}{\partial x_n} & \frac{\partial s_2}{\partial x_n} & \cdots & \frac{\partial s_n}{\partial x_n} \\
		
	\end{bmatrix}
	\begin{bmatrix}
		\frac{\partial g}{\partial s_1} \\
		\frac{\partial g}{\partial s_2} \\
		\vdots \\
		\frac{\partial g}{\partial s_n}
	\end{bmatrix}
\end{equation*}

Para los valores de la jacobiana en sí, estos obtienen de la siguiente ecuación:

\begin{equation*}
	\frac{\partial s_i}{\partial x_j} = \begin{cases}
		s_i (1 - s_i) \quad &\text{si } i = j \\
		- s_i s_j \quad &\text{si } i \neq j \\
	\end{cases}
\end{equation*}

\end{document}